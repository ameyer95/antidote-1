\section{General Abstract Semantics}

Here we will be more general.
Since the concrete semantics manipulate a state triple $\sigma = (T, \varphi, p)$
where $T \in \mathcal{T}$, $\varphi \in \Phi \cup \{ \bot \}$, and $p \in D$,
the abstract semantics will manipulate some abstract state $\sigma^\#$
that is an element in an abstract domain $\mathcal{L}$.
Each abstract element represents some set of concrete elements:
we have the concretization function
$\gamma : \mathcal{L} \rightarrow 2^{\mathcal{T} \times (\Phi \cup \{\bot\}) \times D}$.

We will require that the abstract domain over the program state
support abstract transformers that correspond to the expressions
in the tree-learning-classification DSL,
as well as some particular meets-with-abstractions-of-general-sets.
Given these, we can present abstract semantics for the DSL in more generality
as in Figure~\ref{fig:abstractgeneral}.

\begin{figure}
\centering
\begin{align*}
\den[P]{S ; R}^\#(\mathcal{T}') \coloneqq&~
\den[R]{R}^\#(\den[S]{S}^\#(\alpha(\mathcal{T}' \times \emptyset \times \emptyset))) \\
%
\den[S]{S_1 ; S_2}^\#(\sigma^\#) \coloneqq&~
\den[S]{S_2}^\#(\den[S]{S_1}^\#(\sigma^\#)) \\
%
\den[S]{\iteimpuritynode{S_1}{S_2}}^\#(\sigma^\#) \coloneqq&~
\den[S]{S_1}^\#(\sigma^\# \sqcap \alpha(\{T \in \mathcal{T} : \mathit{impurity}(T) = 0\} \times (\Phi \cup \{\bot\}) \times D)) \\
&~\sqcup
\den[S]{S_2}^\#(\sigma^\# \sqcap \alpha(\{T \in \mathcal{T} : \mathit{impurity}(T) \neq 0\} \times (\Phi \cup \{\bot\}) \times D)) \\
%
\den[S]{\itetrivialnode{S}{U}}^\#(\sigma^\#) \coloneqq&~
\den[S]{S}^\#(\sigma^\# \sqcap \alpha(\mathcal{T} \times \{\bot\} \times D))
\sqcup
\den[U]{U}^\#(\sigma^\# \sqcap \alpha(\mathcal{T} \times \Phi \times D)) \\
%
\den[S]{\bestsplitnode}^\#(\sigma^\#) \coloneqq&~
\mathit{applybestsplit}(\sigma^\#) \\
%
\den[S]{\summarynode}^\#(\sigma^\#) \coloneqq&~
\mathit{applysummary}(\sigma^\#) \\
%
\den[U]{U ; S}^\#(\sigma^\#) \coloneqq&~
\den[S]{S}^\#(\den[U]{U}^\#(\sigma^\#)) \\
%
\den[U]{\itemodelsnode{U_1}{U_2}}^\#(\sigma^\#) \coloneqq&~
\den[U]{U_1}^\#(\sigma^\# \sqcap \alpha(\mathcal{T} \times \{\varphi \in \Phi : x \models \varphi\} \times D)) \\
&~\sqcup
\den[U]{U_2}^\#(\sigma^\# \sqcap \alpha(\mathcal{T} \times \{\varphi \in \Phi : x \not\models \varphi\} \times D)) \\
%
\den[U]{\filternode[(\lnot)]}^\#(\sigma^\#) \coloneqq&~
\mathit{applyfilter}_{(\lnot)}(\sigma^\#) \\
%
\den[R]{\returnnode}^\#(\sigma^\#) \coloneqq&~
\pi_D(\gamma(\sigma^\#))
\end{align*}
\caption{General Abstract Semantics.}
\label{fig:abstractgeneral}
\end{figure}

Accordingly, a given abstract domain $\mathcal{L}$ must implement the following operations:
\begin{itemize}
    \item $\lambda \sigma^\# \ldotp \sigma^\# \sqcap
        \alpha(\{T \in \mathcal{T} : \mathit{impurity}(T) = 0\} \times (\Phi \cup \{\bot\}) \times D)$,
        which should (leave the predicates and posteriors unchanged and)
        try to restrict the possible training sets to those that are ``pure.''
    \item $\lambda \sigma^\# \ldotp \sigma^\# \sqcap
        \alpha(\{T \in \mathcal{T} : \mathit{impurity}(T) \neq 0\} \times (\Phi \cup \{\bot\}) \times D)$,
        which does similarly for mixed-classification training sets.
    \item $\lambda \sigma^\# \ldotp \sigma^\# \sqcap
        \alpha(\mathcal{T} \times \{\bot\} \times D)$,
        which removes all instances of predicates other than the bottom element
        (for descending into the $\varphi = \bot$ if-then-else case).
    \item $\lambda \sigma^\# \ldotp \sigma^\# \sqcap
        \alpha(\mathcal{T} \times \Phi \times D)$,
        which keeps only the non-bottom-element predicates.
    \item $\lambda \sigma^\# \ldotp \sigma^\# \sqcap
        \alpha(\mathcal{T} \times \{\varphi \in \Phi : x \models \varphi\} \times D)$,
        which removes all states in which the predicate does not have $x \models \varphi$.
    \item $\lambda \sigma^\# \ldotp \sigma^\# \sqcap
        \alpha(\mathcal{T} \times \{\varphi \in \Phi : x \not\models \varphi\} \times D)$,
        which behaves similarly for $x \not\models \varphi$.
    \item $\mathit{applybestsplit}(\sigma^\#)$,
        which updates the predicate portion of the represented states
        to correspond to the possible best-splitting-predicate
        based on the possible training sets.
    \item $\mathit{applysummary}(\sigma^\#)$,
        which updates the posterior portion of the represented states
        to correspond to the possible summaries
        based on the possible training sets.
    \item $\mathit{applyfilter}_{(\lnot)}(\sigma^\#)$,
        which updates the training set portion of the represented states
        to correspond to the remainder that satisfy the predicate portion
        (or $\lnot$ of the predicate portion).
\end{itemize}
This is repeated in Figure~\ref{fig:obligations},
which also includes the necessary components for the
specific domains discussed in the remainder of this section.

\begin{figure}
\centering
\begin{tabular}{cl}
Domain & Obligations \\
\toprule
\multirow{9}{*}{General State Domain $\mathcal{L}$} &
$\lambda \sigma^\# \ldotp \sigma^\# \sqcap
\alpha(\{T \in \mathcal{T} : \mathit{impurity}(T) = 0\} \times (\Phi \cup \{\bot\}) \times D)$ \\
& $\lambda \sigma^\# \ldotp \sigma^\# \sqcap
\alpha(\{T \in \mathcal{T} : \mathit{impurity}(T) \neq 0\} \times (\Phi \cup \{\bot\}) \times D)$ \\
& $\lambda \sigma^\# \ldotp \sigma^\# \sqcap
\alpha(\mathcal{T} \times \{\bot\} \times D)$ \\
& $\lambda \sigma^\# \ldotp \sigma^\# \sqcap
\alpha(\mathcal{T} \times \Phi \times D)$ \\
& $\lambda \sigma^\# \ldotp \sigma^\# \sqcap
\alpha(\mathcal{T} \times \{\varphi \in \Phi : x \models \varphi\} \times D)$ \\
& $\lambda \sigma^\# \ldotp \sigma^\# \sqcap
\alpha(\mathcal{T} \times \{\varphi \in \Phi : x \not\models \varphi\} \times D)$ \\
& $\mathit{applybestsplit}(\sigma^\#)$ \\
& $\mathit{applysummary}(\sigma^\#)$ \\
& $\mathit{applyfilter}_{(\lnot)}(\sigma^\#)$ \\
\midrule
\multirow{7}{*}{Box Domain $(\mathcal{L}_\mathcal{T}, \mathcal{L}_{\Phi\cup\{\bot\}}, \mathcal{L}_D)$} &
$\lambda T^\# \ldotp T^\# \sqcap
\alpha(\{T \in \mathcal{T} : \mathit{impurity}(T) = 0\}) :
\mathcal{L}_\mathcal{T} \rightarrow \mathcal{L}_\mathcal{T}$ \\
& $\lambda T^\# \ldotp T^\# \sqcap
\alpha(\{T \in \mathcal{T} : \mathit{impurity}(T) \neq 0\}) :
\mathcal{L}_\mathcal{T} \rightarrow \mathcal{L}_\mathcal{T}$ \\
& $\lambda \varphi^\# \ldotp \varphi^\# \sqcap
\alpha(\{\varphi \in \Phi : x \models \varphi\}) :
\mathcal{L}_{\Phi\cup\{\bot\}} \rightarrow \mathcal{L}_{\Phi\cup\{\bot\}}$\\
& $\lambda \varphi^\# \ldotp \varphi^\# \sqcap
\alpha(\{\varphi \in \Phi : x \not\models \varphi\}) :
\mathcal{L}_{\Phi\cup\{\bot\}} \rightarrow \mathcal{L}_{\Phi\cup\{\bot\}}$\\
& $\mathit{bestsplit}^\#_\Phi : \mathcal{L}_\mathcal{T} \rightarrow
\mathcal{L}_{\Phi \cup \{\bot\}}$ \\
& $\mathit{filter}^\#_{(\lnot)} :
\mathcal{L}_\mathcal{T} \times \mathcal{L}_{\Phi\cup\{\bot\}}
\rightarrow \mathcal{L}_\mathcal{T}$ \\
& $\mathit{summary}^\# : \mathcal{L}_\mathcal{T} \rightarrow \mathcal{L}_D$ \\
\midrule
\multirow{2}{*}{Disjuncts of Boxes Domain $\mathcal{L}^{<\omega}$}
& The underlying box domain $(\mathcal{L}_\mathcal{T}, \mathcal{L}_{\Phi\cup\{\bot\}}, \mathcal{L}_D)$ \\
& $\mathit{filter}^\#_{(\lnot)} :
\mathcal{L}_\mathcal{T} \times \mathcal{L}_{\Phi\cup\{\bot\}} \rightarrow
(\mathcal{L}_\mathcal{T} \times \mathcal{L}_{\Phi\cup\{\bot\}})^*$ \\
\midrule
\multirow{2}{*}{Bounded Disjuncts $\mathcal{L}^{\leq k}$}
& The unbounded disjuncts domain $\mathcal{L}^{<\omega}$ \\
& $\mathit{combine} : \mathcal{L}^{<\omega} \rightarrow \mathcal{L}^{\leq k}$ \\
\bottomrule
\end{tabular}
\caption{Various obligations for abstract domains/transformers.}
\label{fig:obligations}
\end{figure}

\paragraph{Box Domain}
One such abstract domain could be ``boxes'' formed from abstract domains
over each individual element of the state tuple:
if we have some abstractions $T^\#$, $\varphi^\#$, and $p^\#$,
then we could consider a $\sigma^\# = (T^\#, \varphi^\#, p^\#)$.

\begin{figure}
\centering
Where $\sigma^\# = (T^\#, \varphi^\#, p^\#)$,
\begin{align*}
\lambda \sigma^\# \ldotp \sigma^\# \sqcap
\alpha(\{T \in \mathcal{T} : \mathit{impurity}(T) = 0\} \times (\Phi \cup \{\bot\}) \times D)
\coloneqq&~
(T^\# \sqcap_\mathcal{T} \alpha_\mathcal{T}(\{T \in \mathcal{T} : \mathit{impurity}(T) = 0\}),
\varphi^\#, p^\#) \\
%
\lambda \sigma^\# \ldotp \sigma^\# \sqcap
\alpha(\{T \in \mathcal{T} : \mathit{impurity}(T) \neq 0\} \times (\Phi \cup \{\bot\}) \times D)
\coloneqq&~
(T^\# \sqcap_\mathcal{T} \alpha_\mathcal{T}(\{T \in \mathcal{T} : \mathit{impurity}(T) \neq 0\}),
\varphi^\#, p^\#) \\
%
\lambda \sigma^\# \ldotp \sigma^\# \sqcap
\alpha(\mathcal{T} \times \{\bot\} \times D)
\coloneqq&~
(T^\#, \alpha_{\Phi\cup\{\bot\}}(\{\bot\}), p^\#) \\
%
\lambda \sigma^\# \ldotp \sigma^\# \sqcap
\alpha(\mathcal{T} \times \Phi \times D)
\coloneqq&~
(T^\#, \alpha_{\Phi\cup\{\bot\}}(\Phi), p^\#) \\
%
\lambda \sigma^\# \ldotp \sigma^\# \sqcap
\alpha(\mathcal{T} \times \{\varphi \in \Phi : x \models \varphi\} \times D)
\coloneqq&~
(T^\#, \varphi^\# \sqcap_{\Phi\cup\{\bot\}}
\alpha_{\Phi\cup\{\bot\}}(\{\varphi \in \Phi : x \models \varphi\}), p^\#) \\
%
\lambda \sigma^\# \ldotp \sigma^\# \sqcap
\alpha(\mathcal{T} \times \{\varphi \in \Phi : x \not\models \varphi\} \times D)
\coloneqq&~
(T^\#, \varphi^\# \sqcap_{\Phi\cup\{\bot\}}
\alpha_{\Phi\cup\{\bot\}}(\{\varphi \in \Phi : x \not\models \varphi\}), p^\#) \\
%
\mathit{applybestsplit}(\sigma^\#)
\coloneqq&~
(T^\#, \mathit{bestsplit}^\#_\Phi(T^\#), p^\#) \\
%
\mathit{applysummary}(\sigma^\#)
\coloneqq&~
(T^\#, \varphi^\#, \mathit{summary}^\#(T^\#)) \\
%
\mathit{applyfilter}_{(\lnot)}(\sigma^\#)
\coloneqq&~
(\mathit{filter}^\#_{(\lnot)}(T^\#, \varphi^\#), \varphi^\#, p^\#) \\
\end{align*}
\caption{``Boxes'' domain transformers for program state}
\label{fig:boxes}
\end{figure}

Suppose we have some abstract domain over training sets $\mathcal{L}_\mathcal{T}$,
some abstract domain over our predicates (and null value) $\mathcal{L}_{\Phi \cup \{\bot\}}$,
and some abstract domain over our posterior distributions $\mathcal{L}_D$.
Our ``boxes'' abstract domain over states
$\mathcal{L} = (\mathcal{L}_\mathcal{T}, \mathcal{L}_{\Phi \cup \{\bot\}}, \mathcal{L}_D)$
will implement the necessary operations described prior in the largely obvious way:
the details are showin in Figure~\ref{fig:boxes}.
Of course, Figure~\ref{fig:boxes} mostly deligates responsibilities
to functions over its constituent domains.
However, these functions require some interactions between the domains
(instead of being nicely encapsulated within each, individually).
Specifically, we now require:
\begin{itemize}
    \item $\mathcal{L}_\mathcal{T}$ must implement
        $\lambda T^\# \ldotp T^\# \sqcap
        \alpha(\{T \in \mathcal{T} : \mathit{impurity}(T) = 0\})$
        and
        $\lambda T^\# \ldotp T^\# \sqcap
        \alpha(\{T \in \mathcal{T} : \mathit{impurity}(T) \neq 0\})$.
    \item $\mathcal{L}_{\Phi \cup \{\bot\}}$ must implement
        $\lambda \varphi^\# \ldotp \varphi^\# \sqcap
        \alpha(\{\varphi \in \Phi : x \models \varphi\})$
        and
        $\lambda \varphi^\# \ldotp \varphi^\# \sqcap
        \alpha(\{\varphi \in \Phi : x \not\models \varphi\})$.
    \item Joint knowledge of $\mathcal{L}_\mathcal{T}$ and
        $\mathcal{L}_{\Phi \cup \{\bot\}}$ is needed to implement
        $\mathit{bestsplit}^\#_\Phi : \mathcal{L}_\mathcal{T} \rightarrow
        \mathcal{L}_{\Phi \cup \{\bot\}}$ and
        $\mathit{filter}^\#_{(\lnot)} :
        \mathcal{L}_\mathcal{T} \times \mathcal{L}_{\Phi\cup\{\bot\}}
        \rightarrow \mathcal{L}_\mathcal{T}$.
    \item Joint knowledge of $\mathcal{L}_\mathcal{T}$ and $\mathcal{L}_D$
        is needed to implement
        $\mathit{summary}^\# : \mathcal{L}_\mathcal{T} \rightarrow \mathcal{L}_D$.
\end{itemize}
These requirements, too, are included in Figure~\ref{fig:obligations}.
Implementing them for the ``hard-coded'' case presented in the earlier
portions of the document directly follows from the semantics presented in Figure~\ref{fig:abstract}.
Hopefully it is a bit clearer how to handle some increases to expressivity;
for example, to support categorical output,
the only thing that must be updated is
$\mathit{summary}^\# : \mathcal{L}_\mathcal{T} \rightarrow \mathcal{L}_D$
(well, and $\mathcal{L}_D$ itself).

\paragraph{Increasing Precision via Disjuncts}
The main problems with boxes are
\rone that they do not allow us to track joint information
between individual components of the state and
\rtwo that they force us to be very imprecise when combining
largely disjoint sets of concretizations.
This leads to precision loss when, for example,
we perform a join between very different training set abstractions
(e.g.\ during $\mathit{filter^\#}$ when $\gamma(\varphi^\#)$
contains very extensionally different predicates).
We will amend this by
\rone allowing for \emph{disjuncts} of boxes,
which achieves both goals, requiring us to
\rtwo modify the box domain's obligation of $\mathit{applyfilter}$
(and possibly $\mathit{applybestsplit}$ and $\mathit{applysummary}$,
although these are empirically not as urgent for increasing precision)---%
specifially, the underlying $\mathit{filter}^\#$ function
will be allowed to introduce disjuncts.

Accordingly, we will define $\mathcal{L}^{<\omega}$ as
unboundedly many disjuncts of some given box domain
$(\mathcal{L}_\mathcal{T}, \mathcal{L}_{\Phi\cup\{\bot\}}, \mathcal{L}_D)$.
Defining the necessary state domain functions is done in terms of the
lower-level box domain's functions:
if $\sigma^\# \in \mathcal{L}^{<\omega}$ is represented as a finite set
$\sigma^\# = \{\sigma^\#_i : 0 \leq i < n\}$ where each component is in the box domain
(i.e.\ $\sigma^\#_i \in \mathcal{L}$),
then we define all of the necessary $f_{\mathcal{L}^{<\omega}} : \mathcal{L}^{<\omega} \rightarrow \mathcal{L}^{<\omega}$
by applying the lower-level $f_{\mathcal{L}} : \mathcal{L} \rightarrow \mathcal{L}$
to each element in the set,
i.e.\ $f_{\mathcal{L}^{<\omega}}(\sigma^\#) = \{f_{\mathcal{L}}(\sigma^\#_i) : 0 \leq i < n\}$.
The one exception is for $\mathit{applyfilter}$, which instead is defined as:
\[
    \mathit{applyfilter}_{(\lnot)}(\sigma^\#) \coloneqq
    \bigcup_{\sigma^\#_i \in \sigma^\#} (\mathit{filter}^\#_{(\lnot)}(T^\#_i, \varphi^\#_i) \times \{p^\#_i\})
\]
where we now require a new definition of $\mathit{filter}^\#$
that is allowed to introduce additional disjuncts!---%
indeed, the function signature is
$\mathit{filter}^\#_{(\lnot)} : \mathcal{L}_\mathcal{T} \times \mathcal{L}_{\Phi\cup\{\bot\}}
\rightarrow (\mathcal{L}_\mathcal{T} \times \mathcal{L}_{\Phi\cup\{\bot\}})^*$.
This is easy to provide when $\varphi^\#$ is represented as a finite set:
$\mathit{filter}^\#$ then typically involves a large join over all such $\varphi$,
but we simply use the individual elements as a set of disjuncts instead of taking their join.
The join operation for $\mathcal{L}^{<\omega}$ is simple---simply take the union of all the disjuncts.

It should not be hard to imagine that $\mathcal{L}^{<\omega}$ will quickly comprise
a computationally intractable number of disjuncts.
Indeed, it never allows similar disjuncts to merge
(whereas individual boxes always forced dissimilar disjuncts to merge).
Finally, we introduce a domain for boundedly many disjuncts, $\mathcal{L}^{\leq k}$,
which requires only \rone the $\mathit{filter}^\#$ extension needed for $\mathcal{L}^{<\omega}$,
and \rtwo a $\mathit{combine} : \mathcal{L}^{<\omega} \rightarrow \mathcal{L}^{\leq k}$ function
that (perhaps greedily) performs the lower-level box domain join operations
to maintain at most $k$-many disjuncts.
This $\mathit{combine}$ method need only be applied after taking joins
and after $\mathit{applyfilter}$.
