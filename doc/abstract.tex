\section{Abstract Semantics}

Additionally, we provide semantics for calculations on our abstraction $\langle T, n \rangle$,
given in Figure~\ref{fig:abstract}.
Here, the state is a triple $(\langle T, n \rangle, \Phi', p^\#)$
where $\Phi' \subseteq \Phi \cup \{\bot\}$
(because now, since there are multiple possible training sets,
multiple different $\varphi \in \Phi$ could be the best split,
and $\bot$ is for the all-trivial-split possibility)
and where $p^\# \in 2^{[0,1]}$
using some appropriate abstract domain, e.g.\ intervals.
Now, $\den{P}^\# : \langle T, n \rangle \mapsto p^\#$.
Let me point out a few (but not all) details
that introduce imprecision through overapproximations:
\begin{itemize}
    \item In the branching based on $\mathit{impurity}(T) = 0$,
        no refinement of $\mathit{impurity}(T)$ is done when taking the else-branch;
        this is because the dropout domain, as presented,
        is not in general capable of more-tightly expressing an upper bound for
        $\alpha(\{T' \in \langle T, n \rangle \mid \mathit{impurity}(T') \neq 0\})
        \sqsubseteq \langle T, n \rangle$.
    \item When updating the state via
        $\den{\varphi \gets \mathit{bestsplit}_\Phi(T)}^\#$,
        certain $\varphi$ could only be the best split for specific $T' \in \langle T, n \rangle$,
        yet all are incorporated into the single unified abstraction.
        Accordingly, anywhere different dropout sets are joined,
        if their symmetric difference is large, the resultant join
        will be very imprecise.
        (In particular, this happens in $\den{T \gets \mathit{filter}(T, \varphi)}^\#$,
        especially when there exist very extensionally different $\varphi, \psi \in \Phi'$,
        or when $\Phi'$ is simply large.)
    \item When selecting the best split, we completely ignore tie-breaking via $<_\Phi$.
    \item Filtering can produce $\langle T, n \rangle$ with $n = |T|$ (i.e.\ $\emptyset \in \langle T, n \rangle$),
        but we know that if $\varphi$ is bound through selecting a best split,
        in the concrete case, we would never do that if the resultant filter call would equal $\emptyset$.
        The grammar is not strong enough to actaully enforce this, however,
        so instead we allow for $\mathit{summary^\#}(\langle T, n \rangle)$
        to return $\top$ (i.e.\ $[0, 1]$) whenever $n = |T|$ --
        since in the concrete case, we avoid defining $\mathit{summary}$ for $|T| = 0$
        and instead have as an invariant of the semantics that an initially non-empty training set
        will never encounter it.
\end{itemize}

\begin{figure}
\centering
\newcommand{\argtuple}{\langle T, n \rangle, \Phi', p^\#}
\begin{align*}
\den[P]{S ; R}^\#(\langle T, n \rangle) \coloneqq&~
\den[R]{R}^\#(\den[S]{S}^\#(\langle T, n \rangle, \bot, \bot)) \\
%
\den[S]{S_1 ; S_2}^\#(\argtuple) \coloneqq&~
\den[S]{S_2}^\#(\den[S]{S_1}^\#(\argtuple)) \\
%
\den[S]{\iteimpuritynode{S_1}{S_2}}^\#(\argtuple) \coloneqq&~
\text{let } T_0^\# = \mathit{puresets}(\langle T, n \rangle, 0) \text{ and }
T_1^\# = \mathit{puresets}(\langle T, n \rangle, 1) \text{ in } \\
&~\den[S]{S_2}^\#(\langle T, n \rangle, \Phi', p^\#) \\
&~\sqcup
(\text{if } T_0^\# = \bot \text{ then } \bot \text{ else } \den[S]{S_1}^\#(T_0^\#, \Phi', p^\#)) \\
&~\sqcup
(\text{if } T_1^\# = \bot \text{ then } \bot \text{ else } \den[S]{S_1}^\#(T_1^\#, \Phi', p^\#)) \\
%
\den[S]{\itetrivialnode{S}{U}}^\#(\argtuple) \coloneqq&~
(\text{if } \bot \in \Phi' \text{ then } \den[S]{S}^\#(\langle T, n \rangle, \{\bot\}, p^\#) \text{ else } \bot) \\
&~\sqcup
(\text{if } \Phi' \setminus \{\bot\} \neq \emptyset \text{ then }
\den[U]{U}^\#(\langle T, n \rangle, \Phi' \setminus \{\bot\}, p^\#) \text{ else } \bot)\\
%
\den[S]{\bestsplitnode}^\#(\argtuple)
\coloneqq&~ (\langle T, n \rangle, \mathit{bestsplit}_\Phi^\#(\langle T, n \rangle), p^\#) \\
%
\den[S]{\summarynode}^\#(\argtuple) \coloneqq&~
(\langle T, n \rangle, \Phi', \mathit{summary}^\#(\langle T, n \rangle)) \\
%
\den[U]{S}^\#(\argtuple) \coloneqq&~
\den[S]{S}^\#(\argtuple) \\
%
\den[U]{U_1 ; U_2}(\argtuple) \coloneqq&~
\den[U]{U_2}^\#(\den[U]{U_1}^\#(\argtuple)) \\
%
\den[U]{\itemodelsnode{U_1}{U_2}}^\#(\argtuple) \coloneqq&~
\text{let } \Psi = \{\varphi \in \Phi' : x \models \varphi\} \text{ in } \\
&~(\text{if } \Psi = \emptyset \text{ then } \bot \text{ else }
\den[U]{U_1}^\#(\langle T, n \rangle, \Psi, p^\#)) \\
&~\sqcup
(\text{if } \Phi' \setminus \Psi = \emptyset \text{ then } \bot \text{ else }
\den[U]{U_2}^\#(\langle T, n \rangle, \Phi' \setminus \Psi, p^\#)) \\
%
\den[U]{\filternode[(\lnot)]}^\#(\argtuple) \coloneqq&~
(\bigsqcup_{\varphi \in \Phi'} (\langle T, n \rangle \upharpoonright (\lnot) \varphi), \Phi', p^\#) \\
%
\den[R]{\returnnode}^\#(\argtuple) \coloneqq&~ p^\#
\end{align*}
where
\begin{align*}
(\langle T_1, n_1 \rangle, \Phi_1', p_1^\#) \sqcup (\langle T_2, n_2 \rangle, \Phi_2', p_2^\#)
\coloneqq&~
(\langle T_1, n_1 \rangle \sqcup \langle T_2, n_2 \rangle,
\Phi_1' \cup \Phi_2', p_1^\# \sqcup p_2^\#) \\
%
T \upharpoonright \varphi \coloneqq&~ \{(x,y) \in T \mid x \models \varphi\} \\
%
\langle T, n \rangle \upharpoonright \varphi \coloneqq&~
\langle T \upharpoonright \varphi, \text{min}(n, |T \upharpoonright \varphi|) \rangle \\
%
\mathit{puresets}(\langle T, n\rangle, \hat{y}) \coloneqq&~
\text{let } T' = \{(x,y) \in T \mid y = \hat{y}\} \text{ in }
\text{if } |T \setminus T'| \leq n \text{ then } \langle T', n - |T \setminus T'| \rangle
\text{ else } \bot \\
%
\mathit{summary}^\#(\langle T, n \rangle) \coloneqq&~
\text{if } n = |T| \text{ then } [0, 1] \text{ else }
\text{let } c_1 = |\{(x,y) \in T : y = 1\}| \text{ in }
\frac{[max(0, c_1 - n), c_1]}{[|T| - n, |T|]}\\
%
\mathit{impurity}^\#(\langle T, n \rangle) \coloneqq&~
[2,2] \cdot \mathit{summary^\#(\langle T, n \rangle)}
\cdot ([1,1] - summary^\#(\langle T, n \rangle)) \\
%
\mathit{score}^\#(\langle T, n \rangle, \varphi) \coloneqq&~
\text{let } \langle T_0, n_0 \rangle = \langle T, n \rangle \upharpoonright \lnot \varphi
\text{ and } \langle T_1, n_1 \rangle = \langle T, n \rangle \upharpoonright \varphi \text{ in } \\
&~[|T_0| - n_0, |T_0|] \cdot \mathit{impurity}^\#(\langle T_0, n_0 \rangle) +
[|T_1| - n_1, |T_1|] \cdot \mathit{impurity}^\#(\langle T_1, n_1 \rangle)\\
%
\mathit{bestsplit}^\#_\Phi(\langle T, n \rangle) \coloneqq&~
\text{let } \Psi = \{\varphi \in \Phi : \langle T, n \rangle \upharpoonright \varphi \neq \bot
\land \langle T, n \rangle \upharpoonright \lnot\varphi \neq \bot\} \text{ in } \\
&~\text{let } \tau = \min_{\varphi^* \in \Psi} \max(\mathit{score}^\#(\langle T, n \rangle, \varphi^*)) \text{ in } \\
&~\{\varphi \in \Psi : \min(\mathit{score}^\#(\langle T, n \rangle, \varphi)) \leq \tau\}\\
&~\cup (\text{if } \forall \varphi \in \Psi \ldotp \bot \in \langle T, n \rangle \upharpoonright \varphi
\lor \bot \in \langle T, n \rangle \upharpoonright \lnot\varphi \text{ then } \{\bot\} \text{ else } \emptyset) \\
\end{align*}
\caption{Dropout Semantics (note that many of the auxiliary functions involve interval arithmetic).}
\label{fig:abstract}
\end{figure}


Again, we have that $\bot \not \in \Phi'$ when we need to use it.
\begin{proposition}[todo]
\todo{Write the analog of the concrete one}
\end{proposition}

And, perhaps most importantly, the semantics are sound.
\begin{theorem}[Soundness]
If $T' \in \langle T, n \rangle$ then $\den{P}(T') \in \den{P}^\#(\langle T, n \rangle)$.
\end{theorem}
\begin{proof}
``Trivial,'' ``obvious,'' etc.
\todo{Actually prove things.}
\end{proof}
