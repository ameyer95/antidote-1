\section{Concrete Semantics}

We give concrete semantics for evaluating instantiations of the language
in Figure~\ref{fig:concrete};
built into these semantics are choices for parameters of decision tree learning---%
notably, $\mathit{impurity(T)}$ uses the \emph{Gini impurity} criteria
(a common alternative to Shannon entropy, used for its mathematical simplicity
/ avoidance of logarithms).
The high-level usage involves the semantics of a program $P \coloneqq S ; R$
with $\den{P} : T \mapsto p$.
$\den{\cdot}$ is defined with respect to the fixed test input $x$ and set of predicates,
so $\den{P}$ maps a training set to the posterior distribution describing $x$'s classification
(specifically, a $[0,1]$-valued Bernoulli parameter.)
In the interest of formality, Figure~\ref{fig:concrete} writes the definition of $\den{\cdot}$
using $\den[P]{\cdot}$, $\den[S]{\cdot}$, $\den[U]{\cdot}$, and $\den[R]{\cdot}$.

\begin{figure}
\centering
Globals: test input $x \in \mathcal{X}$ and totally-ordered predicates $(\Phi, <_\Phi)$
\begin{align*}
\den[P]{S ; R}(T) \coloneqq&~
\den[R]{R}(\den[S]{S}(T, \bot, \bot)) \\
%
\den[S]{S_1 ; S_2}(T, \varphi, p) \coloneqq&~
\den[S]{S_2}(\den[S]{S_1}(T, \varphi, p)) \\
%
\den[S]{\iteimpuritynode{S_1}{S_2}}(T, \varphi, p) \coloneqq&~
\text{if } \mathit{impurity}(T) = 0 \text{ then } \den[S]{S_1}(T, \varphi, p)
\text{ else } \den[S]{S_2}(T, \varphi, p) \\
%
\den[S]{\itetrivialnode{S}{U}}(T, \varphi, p) \coloneqq&~
\text{if } \varphi = \bot \text{ then } \den[S]{S}(T, \varphi, p)
\text{ else } \den[U]{U}(T, \varphi, p) \\
%
\den[S]{\bestsplitnode}(T, \varphi, p) \coloneqq&~
(T, \mathit{bestsplit}_\Phi(T), p) \\
%
\den[S]{\summarynode}(T, \varphi, p) \coloneqq&~
(T, \varphi, \mathit{summary}(T)) \\
%
\den[U]{U ; S}(T, \varphi, p) \coloneqq&~
\den[S]{S}(\den[U]{U}(T, \varphi, p)) \\
%
\den[U]{\itemodelsnode{U_1}{U_2}}(T, \varphi, p) \coloneqq&~
\text{if } x \models \varphi \text{ then } \den[U]{U_1}(T, \varphi, p)
\text{ else } \den[U]{U_2}(T, \varphi, p) \\
%
\den[U]{\filternode[(\lnot)]}(T, \varphi, p) \coloneqq&~
(T \upharpoonright (\lnot)\varphi, \varphi, p) \\
%
\den[R]{\returnnode}(T, \varphi, p) \coloneqq&~ p
\end{align*}
where $T \subseteq \mathcal{X} \times \{0,1\}$ and
\begin{align*}
T \upharpoonright \varphi \coloneqq&~ \{(x,y) \in T \mid x \models \varphi\} \\
\mathit{summary}(T) \coloneqq&~ \frac{|\{(x,y) \in T : y = 1\}|}{|T|} \\
\mathit{impurity}(T) \coloneqq&~ 2 \cdot \mathit{summary}(T) \cdot (1-\mathit{summary}(T)) \\
\mathit{score}(T, \varphi) \coloneqq&~
|T \upharpoonright \varphi| \cdot \mathit{impurity}(T \upharpoonright \varphi) +
|T \upharpoonright \lnot\varphi| \cdot \mathit{impurity}(T \upharpoonright \lnot\varphi) \\
\mathit{bestsplit}_\Phi(T) \coloneqq&~
\text{let } \Phi' = \{\varphi \in \Phi : T \upharpoonright \varphi \not\in \{\emptyset, T\}\} \text{ in } \\
&~\text{if } \Phi' = \emptyset \text{ then } \bot \text{ else }
\text{min}_{<_\Phi} \text{argsmin}_{\varphi^* \in \Phi'}~\mathit{score}(T, \varphi^*)
\end{align*}
\caption{Concrete Semantics.
Note in the definition of $\mathit{bestsplit}_\Phi(T)$
that I use ``argsmin'' to denote the \emph{set} of arguments
that minimize the objective (to be rigorous about ties).}
\label{fig:concrete}
\end{figure}

\begin{proposition}[Well-defined concrete semantics]
If $P$ is a program generated from the grammar in Figure~\ref{fig:dsl}, then
\begin{itemize}
    \item If $T \neq \emptyset$, then $\den{P}(T)$ never divides by zero
        (as might seem possible in $\mathit{summary}(T)$).
    \item $\den{P}$ never involves a computation of
        $T \upharpoonright \varphi$, $x \models \varphi$, or $\mathit{score}(T, \varphi)$
        in which $\varphi = \bot$.
\end{itemize}
\end{proposition}
